%
% Created by Oliver Erxleben
% Copyright (c) 2011 - 2012 . Provided as is :)
%
\documentclass[toc=bib,toc=listof, 12pt]{scrreprt}
\AtBeginDocument{\addtocontents{toc}{\protect\thispagestyle{empty}}} 
\usepackage[ngerman]{babel} 
\usepackage[utf8]{inputenc}
\usepackage[scaled]{helvet}
\renewcommand*\familydefault{\sfdefault} %% Only if the base font of the document is to be sans serif
\usepackage[T1]{fontenc}

\documentclass[border=10]{standalone}
\usepackage{tikz}

% Directory Tree settings 
\makeatletter
\newcount\dirtree@lvl
\newcount\dirtree@plvl
\newcount\dirtree@clvl
\def\dirtree@growth{%
  \ifnum\tikznumberofcurrentchild=1\relax
  \global\advance\dirtree@plvl by 1
  \expandafter\xdef\csname dirtree@p@\the\dirtree@plvl\endcsname{\the\dirtree@lvl}
  \fi
  \global\advance\dirtree@lvl by 1\relax
  \dirtree@clvl=\dirtree@lvl
  \advance\dirtree@clvl by -\csname dirtree@p@\the\dirtree@plvl\endcsname
  \pgf@xa=0.5cm\relax % horizontal margin
  \pgf@ya=-0.75cm\relax % vertical margin
  \pgf@ya=\dirtree@clvl\pgf@ya
  \pgftransformshift{\pgfqpoint{\the\pgf@xa}{\the\pgf@ya}}%
  \ifnum\tikznumberofcurrentchild=\tikznumberofchildren
  \global\advance\dirtree@plvl by -1
  \fi
}

\tikzset{
  dirtree/.style={
    growth function=\dirtree@growth,
    every node/.style={anchor=north},
    every child node/.style={anchor=west},
    edge from parent path={(\tikzparentnode\tikzparentanchor) |- (\tikzchildnode\tikzchildanchor)}
  }
}

\makeatother

\begin{document}

	\begin{tikzpicture}[dirtree]
	\node {toplevel} 
	    child { node {Foo}
	            child { node {foo} }
	            child { node {bar} }
	            child { node {baz} }
	    }
	    child { node {Bar}
	        child { node {foo} }
	        child { node {foo} }
	        child { node {foo} }
	        child { node {foo} }
	        child { node {bar} }
	        child { node {baz} }
	    }
	    child { node {Baz}
	        child { node {foo} }
	        child { node {bar} }
	        child { node {baz} }
	    };
	\end{tikzpicture}

\end{document}
%--- Abstände, Ränder usw.
\usepackage{geometry}
\geometry{verbose,a4paper,
	tmargin=2.5cm,bmargin=3cm,lmargin=3cm,rmargin=3cm}
\usepackage{hyperref}
\hypersetup{
  ,colorlinks=true
  ,linkcolor=black
  ,citecolor=black
  ,filecolor=black
  ,urlcolor=black
}
\usepackage{float}
\usepackage[absolute]{textpos} 
\usepackage[headsepline,footsepline]{scrpage2}
\usepackage{listings}
\usepackage{caption}
\DeclareCaptionFont{white}{\color{white}}
\DeclareCaptionFormat{listing}{\colorbox{gray}{\parbox{\textwidth}{#1#2#3}}}
\captionsetup[lstlisting]{format=listing,labelfont=white,textfont=white}
\usepackage{ifpdf}
\ifpdf
\usepackage[pdftex]{graphicx}
\else
\usepackage{graphicx}
\fi
%% Define Pagestyle header and footer
\pagestyle{scrheadings}
\automark[section]{chapter}
\ihead{\ \\\leftmark} \chead{} \ohead{}
\ifoot{\ebtstudentname} \cfoot{} \ofoot{\pagemark}
\input{mycommands}
\renewcommand{\chapterpagestyle}{scrheadings}
\renewcommand*{\chapterheadstartvskip}{}

% Titlepage settings
\input{packages/title}

\begin{document}
\selectlanguage{ngerman}

% Titleblatt einfügen
\ThisCenterWallPaper{1}{images/thesisdeckblatt.jpg}
\maketitle
% Titelblatt einfügen (end)
% Vorwort
\pagestyle{empty}
\input{chapters/vorwort.tex}
% Vortwort (end)
\newpage
\input{chapters/notation}
\newpage
\tableofcontents
% Zusammenfassung
\newpage
\clearpage
\thispagestyle{empty}
\pagenumbering{Roman} 

%\addchap{Zusammenfassung} % (fold)
%\label{cha:zusammenfassung}
%\pagestyle{empty}
% Zusammenfassung (end)

% Beginn Thesis
\newpage
%\input{chapters/einleitung_start}
\pagenumbering{arabic}
\pagestyle{scrheadings}
\chapter{Einleitung} % (fold)
\setcounter{page}{1}
\label{sec:einleitung}
	\section{Motivation} % (fold)
	\label{sec:motivation}
		\input{chapters/motivation}
	\pagebreak
	\section{Vorgehen und Ablauf} % (fold)
	\label{sec:vorgehen_und_ablauf}
		\input{chapters/vorgehen_und_ablauf}			
	\section{Betriebsmittel} 
	\label{sec:betriebsmittel}
		\input{chapters/betriebsmittel}
% chapter einleitung (end)
\newpage
\clearpage
\thispagestyle{empty}
\input{chapters/ziele_start}
\chapter{Ziele}
\setcounter{page}{4}
\label{sec:ziele}	
	\section{Hintergrund}
	\label{sec:hintergrund}
		\input{chapters/hintergrund}
		\pagebreak
	\section{Problembeschreibung}
		\input{chapters/problembeschreibung}
	\section{Aufgabenstellung}
		\input{chapters/aufgabenstellung}
	%\section{Erwartete Ergebnisse}
	%	\input{chapters/erwartete_ergebnisse}
\newpage
\clearpage
\thispagestyle{empty}
\input{chapters/grundlagen_start}	
\chapter{Grundlagen}
\setcounter{page}{6}
\label{sec:grundlagen}
	\section{Begriffsdefinitionen}
	\label{sec:begriffsdefinitionen}
		\subsection{Verteilte Systeme}
			\input{chapters/verteilte_systeme}
		\subsection{Kollaborative Systeme}
			\input{chapters/kollaborative_systeme}
	%	\subsection{HTML}
	%		\input{chapters/html.tex}
% (19.02.2012) Auskommentiert: Liegt nun in Kommunikation im Web
%		\subsection{HTTP-Protokolle und -Server}
%			\input{chapters/http_https}	
	\section{Web-Anwendungen}
	\label{sec:web_anwendungen}
		\input{chapters/web_apps}
	\section{Kommunikation im Web}
	\label{sec:kommunikation_im_web}
		\input{chapters/kommunikation_im_web}
%	\section{Formen der Datenübertragung}
%	\label{sec:datenuebertragung}
%		\input{chapters/datenuebertragung}
\newpage
\clearpage
\thispagestyle{empty}
\input{chapters/websockets_start}				
\chapter{Web-Sockets}
\setcounter{page}{16}
	\section{Sockets}
	\label{sec:sockets}
		\input{chapters/sockets}
		\pagebreak
	\section{Web-Sockets}
		\input{chapters/websockets}
\newpage
\clearpage
\thispagestyle{empty}
\input{chapters/konzept_impl_start}		
\chapter{Konzeption und Implementierung}
\label{sec:konzeption}
\setcounter{page}{24}
	%\input{chapters/impl_einfuehrung}
	\section{Entwicklungsumgebung}
		\input{chapters/entwicklungsumgebung}
	\section{Projektstruktur}
		\input{chapters/projektstruktur}
	\section{Websocket-Server}
		\input{chapters/server_interface}
		\input{chapters/server_class}
		\input{chapters/socket_daemon}
	\section{Datenaustausch}
		\input{chapters/datenaustausch}
	\section{Beispielimplementierung}
		\subsection{Idee}
			\input{chapters/idea}
			\input{chapters/ui}
		\subsection{Verwendete Webtechniken im Client}
				\input{chapters/programmiersprachen}
		\subsection{Zeichenfunktion}
			\input{chapters/zeichenfunktion}
		\subsection{Verbindung zum Server}
			\input{chapters/client_server_connect}
			\input{chapters/ws_interface}
	\pagebreak
\newpage
\clearpage
\thispagestyle{empty}
\input{chapters/fazit_start}
\chapter{Fazit} % (fold)
\label{sec:fazit}
\setcounter{page}{48}
\input{chapters/fazit}

% Abkürzungsverzeichnis
\addchap{Abkürzungsverzeichnis} % (fold)
\pagenumbering{Roman}
\setcounter{page}{2}
\label{sec:abkuerzungsverzeichnis}
\begin{description}  
  \input{chapters/abkuerzungsverzeichnis.tex} 
\end{description}

\listoffigures

\listoftables
\lstlistoflistings
% Literaturverzeichnis
\begin{appendix}
\input{chapters/literaturverzeichnis}
\input{chapters/anlagenverzeichnis}
\end{appendix}

\end{document}